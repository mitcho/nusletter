\documentclass[a4paper]{article}

\usepackage{nusletter}

\usepackage{lipsum}

\begin{document}

\nushead{\textbf{Faculty of Arts and Social Sciences}\\
\small Department of English, Linguistics and Theatre Studies}

\nusfoot{Block AS5, 7 Arts Link, Singapore 117570\\
Tel: (65) 6516 3914 / 3915 / 3917 \hskip 5pt Fax: (65) 6773 2981\\
Website: fass.nus.edu.sg/elts}

\pagestyle{empty}

\noindent\today

\

\noindent Dear fellow NUS staff,

The \verb`nusletter` \LaTeX{} package generates a first page which resembles NUS letterhead. To use, add \verb`\usepackage{nusletter}` to your preamble. If you prefer a blue header, add the package option \verb`blue`. If you prefer a black and white header, which may be printed without edge-to-edge printing, add the package option \verb`black`. Use the \verb`\nushead{}` and \verb`\nusfoot{}` commands in your document body, right after \verb`\begin{document}`, to add text to the header and footer. (The Source Sans Pro font is used for this purpose, as the official Frutiger font is not freely available.)

\lipsum

\lipsum

\lipsum

\hspace*{3.5in}\begin{minipage}{3in}
\-\\Sincerely,\\
\\
Michael Yoshitaka Erlewine (mitcho)\\
Associate Professor, NUS\\
\texttt{mitcho@nus.edu.sg}
\end{minipage}

\end{document}
